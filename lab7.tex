\documentclass{report}
\usepackage{amsmath,amsthm,amssymb}
\usepackage{mathtext}
\usepackage[T1,T2A]{fontenc}
\usepackage[utf8]{inputenc}
\usepackage[russian, english]{babel}
\usepackage{graphicx}

\begin{document}


\section*{Лабораторная работа №7. Задача Коши}
Выполнил студент группы 429 Сурмин евгений Олегович
\section*{Вариант №2}
Решить методом Тейлора 3-го порядка задачу Коши $$y'' + 4y = \sin(x), \quad y(0) = 1, y'(0) = 0, \quad x \in [0, 2]$$ с заданной относительной точностью 0.01. Требуется построение графиков решения $y(x), y'(x)$, а также фазовых траекторий. 
\section*{Теоретическая часть}
Простейшим способом построения приближенного решения задачи Коши в точке $x_{n+1}$ сетки является способ, основанный на разложении решения в ряд Тейлора в предыдущей точке сетки $x_n$ по степеням шага $h$: $$y(x_{n+1}) = y(x_n) + h \varphi_p (x_n, y_n, h),$$ $$\varphi_p (x_n, y_n, h) = y'(x) + \cfrac{h}{2} y''(x) + \dots + \cfrac{h^{p-1}}{p!} y^{(p)}(x)$$При $p = 3$ получаем метод Тейлора 3-го порядка. Для каждой $i$-ой итерации будем искать $$\begin{cases} y_{i+1} = y_i + h y'_i + \cfrac{h^2}{2!}y''_i + \cfrac{h^3}{3!}y'''_i \\
y'_{i+1} = y'_i + hy''_i + \cfrac{h^2}{2!}y'''_i + \cfrac{h^3}{3!}y^{(4)}_i \\
y''_{i+1} = \sin(x) - 4y_i \\
y'''_{i+1} = \cos(x) - 4y'_i \\
y^{(4)}_{i+1} = -\sin(x)- 4y''_i
\end{cases}$$ Шаг $h$ определяется из следующих соображений. Для начала положим $h = h_0$ и найдем $y_h = (y_0, y_1, \dots, y_{n-1})$ и $y'_h = (y'_0, y'_1, \dots, y'_{n-1})$. Затем удвоим шаг и найдем $y_{2h}$ и $y'_{2h}$. Будем повторять эти действия пока $\max(y_h - y_{2h}) \geq \varepsilon$ и $\max(y'_h - y'_{2h}) \geq \varepsilon$.
\section*{Практическая часть}
Функция void metod\_teylora(double h, double *y, double *dy) заполняет массивы y и dy найденным решением, функция double opt\_h(double eps) возвращает оптимальный шаг. Функция int main() выводит на экран величину шага h и записывает решение в файл resh.txt.
\section*{Результаты}
В результате работы программы было найденно решение задачи Коши. Оно представленно на рисунках ниже, где также проиллюстрирован фазовый портрет.

\begin{figure}[h]
    \centering
    \includegraphics[width=12cm][1](img1.png)
\end{figure}

\begin{figure}[h]
    \centering
    \includegraphics[width=12cm][2](img2.png)
\end{figure}
\end{document}
    
